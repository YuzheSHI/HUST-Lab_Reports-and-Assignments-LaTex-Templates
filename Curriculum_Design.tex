\documentclass{ctexrep}
\usepackage[margin=25mm]{geometry}
\usepackage{amsmath}
\usepackage{fancyhdr}
%\usepackage[linesnumbered,boxed]{algorithm2e}
\pagestyle{fancy}
\chead{华中科技大学计算机科学与技术学院}


\begin{titlepage}
    \title{Curriculum Design}
    \author{Yuzhe SHI}
    \date{Feburary 2020}
\end{titlepage}


\begin{document}
\maketitle
\bibliographystyle{plain}

  
\begin{abstract}
   这是课程设计。 
\end{abstract}
\chapter{任务书}
\section{设计内容}
SAT问题即命题逻辑公式的可满足性问题\textit{(satisfiability problem)},是计算机科学与人工智能基本问题,是一个典型的NP完全问题,可广泛应用于许多实际问题如硬件设计、安全协议验证等,具有重要理论意义与应用价值。本设计要求基于DPLL算法实现一个完备SAT求解器,对输入的CNF范式算例文件,解析并建立其内部表示;精心设计问题中变元、文字、子句、公式等有效的物理存储结构以及一定的分支变元处理策略,使求解器具有优化的执行性能;对一定规模的算例能有效求解,输出与文件保存求解结果,统计求解时间。    
\section{设计要求}
\subsection{输入输出功能}
包括程序执行参数的输入,SAT算例cnf文件的读取,执行结果的输出与文件保存等。(15\%)
\subsection{公式解析与验证}
读取cnf算例文件,解析文件,基于一定的物理结构,建立公式的内部表示;并实现对解析正确性的验证功能,即遍历内部结构逐行输出与显示每个子句,与输入算例对比可人工判断解析功能的正确性。数据结构的设计可参考文献[1-3]。(15\%)
\subsection{DPLL过程}
基于DPLL算法框架,实现SAT算例的求解。(35\%)
\subsection{时间性能的测量}
基于相应的时间处理函数(参考time.h),记录DPLL过程执行时间(以毫秒为单位),并作为输出信息的一部分。(5\%)
\subsection{程序优化}
对基本DPLL的实现进行存储结构、分支变元选取策略[1-3]等某一方面进行优化设计与实现,提供较明确的性能优化率结果。优化率的计算公式为:
\begin{equation}
    \frac{t-t_0}{t_0}\times 100\%
\end{equation}
其中$t$为未对DPLL优化时求解基准算例的执行时间,$t_o$则为优化DPLL实现时求解同一算例的执行时间。(15\%)
\subsection{SAT应用}
将二进制数独游戏问题转化为SAT问题,并集成到上面的求解器进行问题求解,游戏可玩,具有一定的简单的交互性。

\tableofcontents  


\end{document}