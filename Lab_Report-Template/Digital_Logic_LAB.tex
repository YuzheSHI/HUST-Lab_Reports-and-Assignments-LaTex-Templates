\documentclass{ctexrep}
\usepackage[margin=20mm]{geometry}
\usepackage{amsmath}
\usepackage{fancyhdr}
\usepackage{algorithm2e}
\usepackage{graphicx}
\pagestyle{fancy}
\chead{数字逻辑实验报告}
\title{Digital Logic Lab}
\author{Yu-Zhe Shi}

\begin{document}
\maketitle
\begin{abstract}
    这是数字逻辑实验课程报告。
\end{abstract}
\tableofcontents
\chapter{系列二进制加法器设计}
\section{实验方案设计}
\subsection{逻辑描述}
\subsubsection{一位二进制半加器}
\begin{itemize}
    \item 输入:被加数A,加数B。
    \item 输出:本位和S,进位信号C。
    \item 分析:根据一位加法法则,只有加数都是1时才进位,否则进位信号为零。进位时本位和为0. 因此,使用异或运算计算本位和,与运算计算进位信号。
    \item 逻辑公式:\begin{equation}
        S=A\oplus B,\;C=A\times B
    \end{equation}
\end{itemize}
\subsection{电路图}
\begin{figure}[h]
    %\centering\includegraphics[width=16cm]{}\caption{一位二进制半加器}\label{fig 1}
\end{figure}

\chapter{小型实验室门禁系统设计}


\chapter{无符号数的乘法器设计}


\chapter{无符号数的除法器设计}


\chapter{多功能电子钟系统设计}


\chapter{斐波那契(Fibonacci)数列计算器设计}


\chapter{实验课总结}





\end{document}
